

\title{Project Step 1 : Project and Database Outline\\[.25cm]
						\textbf{Open Resource Learning Database}}
\author{Marc Tibbs (tibbsm@oregonstate.edu)\\
        CS340-Spring 2018}
\date{\today}

\documentclass[12pt]{article}
\usepackage{pdfpages}

\begin{document}
\maketitle

% \begin{abstract}
% \end{abstract}

\section{Project Outline}
I plan on building a database to be used by students who are interested in learning new subjects with other people. The database will collect data from easily obtainable educational resources and keep track of students, groups of students, and the projects that they are working on. There is a rapidly growing amount of educational resources on the internet available to the masses so there is plenty of data to add to the database. This database will provide a service to many people who are looking to learn a new skill and socializing.

\section{Database Outline}

The entities in the database are:

\begin{itemize}
	\item \textbf{Student} -- This entity represents students within the database. Students are able to join classes and project groups. It has the following attributes:
	\begin{itemize}
		\item \textbf{stu\_id:} This number is automatically assigned to a student when they are recorded in the database. It is an auto-incrementing number which is the primary key. 
		\item \textbf{stu\_fname:} This attribute represents the student's first name and is a string of at most 100 characters. It cannot be blank and there is no default.
		\item \textbf{stu\_lname:} This attribute represents the student's last name and is a string of at most 100 characters. It cannot be blank and there is no default.
		\item \textbf{stu\_email:} This attribute represents the student's email address and is a string of at most 100 characters. It cannot be blank and there is no default.
	\end{itemize}

	\item \textbf{Class} -- This entity represents classes that students can participate in. Each class has its own project group in which students can build something using what they learn in a specific class. It has the following attributes:
	\begin{itemize}
		\item \textbf{cla\_id:} This number is automatically assigned to a class when it is recorded in the database. It is an auto-incrementing number which is the primary key. 
		\item \textbf{cla\_title:} This attribute represents the official title of a class. It is a string of at most 100 characters. It cannot be blank and there is no default.
		\item \textbf{cla\_url:} This attribute represents the web address of the class's homepage. It is a string of at most 100 characters. It cannot be blank and there is no default. 
	\end{itemize}

	\item \textbf{Group} -- Each project group is composed of many students and belongs to one class. It has the following attributes:
	\begin{itemize}
		\item \textbf{gro\_id:} This number is automatically assigned to a group when it is recorded in the database. It is an auto-incrementing number which is the primary key. 
		\item \textbf{gro\_name:} This attribute represents the name of a group. It is a string of no more than 100 characters. It cannot be blank and there is no default. 
	\end{itemize}

	\item \textbf{Resource} -- Each class has additional resources which students can refer to. It has the following attributes:
	\begin{itemize}
		\item \textbf{res\_id:} This number is automatically assigned to a resource when it is recorded in the database. It is an auto-incrementing number which is the primary key.
		\item \textbf{res\_title:} This attribute represents the title of the resource. It is a string of no more than 100 characters. It cannot be blank and there is no default. 
		\item \textbf{res\_author:} This attribute represents the author of the resource. It is a string of no more than 100 characters. It can be left blank and there is no default. 
		\item \textbf{res\_url:} This attribute represents the web address where the resource might be available. It is a string of no more than 100 characters. It can be left blank and there is no default. 
	\end{itemize}
\end{itemize}

The relationships in the database are:
\begin{itemize}
	\item \textbf{Students can take classes.} A student can participate in multiple classes and a class can have multiple students so the relationship is a many-to-many relationship.
	\item \textbf{Students can participate in project groups.} A student can participate in multiple project groups and a group can have multiple students. This represents  a many-to-many relationship.
	\item \textbf{Classes can have additional resources.} Each class may have multiple additional resources for students to learn from. Additionally, each resource can have a relationship with multiple classes. This represents another many-to-many relationship.
	\item \textbf{Each class has its own group.} Each class has a group that will learn together and work on a group project together. Each class can only have one group and each group is related to only one class. This represents a one-to-one relationship.

\end{itemize}

\includepdf[pages={1}]{DB.pdf}


\end{document}